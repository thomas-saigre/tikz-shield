%!TeX root=../tikz-shield-doc.tex
\section{Basic usage}

A badge can be included using the following command, in which the first argument specifies the left text (\textsf{Any text} in the example above), and the second argument is the text on the right (\textsf{you like}).

\begin{tcblisting}{title={\tikzshield}, lower separated=false}
\drawBadge{Any text}{you like}
\end{tcblisting}

The color of the right part can be customized, the default color being a rectangle of color \showColor{default-blue}.


\begin{tcblisting}{title={Change color}, lower separated=false}
\drawBadge[color=red]{Any text}{you like}
\drawBadge[labelColor=white, colorLeft=red]{Any text}{you like}
\drawBadge[labelColor=pink, color=black, colorRight=yellow]{Any text}{you like}
\end{tcblisting}


\begin{tcblisting}{title={Customize logo}, lower separated=false}
\drawBadge[logo=\faFilm]{Movie}{5~\faStar}
\drawBadge[logoColor=black, logo=\faHistory, labelColor=yellow]{Any text}{you like}
\end{tcblisting}


\section{Optional arguments}

Some optional arguments can be passed to the macro \texttt{\textbackslash{}drawBadge} (as well as the other ones described in this document):

\begin{center}
    \begin{tblr}{
        colspec = {Q[c, cmd=\texttt]lcc},
        row{1} = {cmd=\bf, bg=default-blue, fg=white},
        vlines, hlines,
    }
    Name & Description & Default value & Pkg \\
    labelColor & Background color of the left part of the badge & \showColor{default-left} & \checkmark \\
    color & Background color of the right part of the badge & \showColor{default-blue} & \checkmark \\
    colorLeft & Text color of the right part of the badge & \showColor{white} & \checkmark \\
    colorRight & Text color of the right part of the badge & \showColor{white} & \checkmark \\
    logo & Logo displayed before the text & & \\
    logoColor & Color of the logo & \showColor{white} & \checkmark \\
    \end{tblr}
\end{center}

If the last column is ticked, then the option can be specified as default option for the whole library, that can be passed when loading it, for instance
\mintinline[breaklines]{tex}{\usepackage[labelColor=red]{tikz-shield}}.



\section{Special badges}


Some presets are available :

\begin{tcblisting}{title={Github badge}, lower separated=false}
\githubBadge{thomas-saigre/tikz-shield}
\end{tcblisting}

\begin{tcblisting}{title={Gitlab badge}, lower separated=false}
\gitlabBadge[color=orange]{inkscape/inkscape}
\end{tcblisting}

\begin{tcblisting}{title={Custom forge badge}, lower separated=false}
\forgeBadge[labelColor=black, logo=\faCode]{code.videolan.org/videolan/vlc}
\forgeBadge[labelColor=green, logo=\checkmark]{code.videolan.org/videolan/vlc}
\end{tcblisting}

In all cases, the generated badge is clickable and redirect to the repository provided.
The icon can be customized, as well as the color.
Note that anything can be used as a logo, even a text or a personalized command that you would have created (in previous example
\mintinline[breaklines]{tex}{\def\checkmark{\tikz\fill[scale=0.4] (0,.35) -- (.25,0) -- (1,.7) -- (.25,.15) -- cycle;}}).
